\section{\red{Implementation and Results}}


%\subsubsection{Implementation}
We implemented a two module prototype to showcase our method. The \texttt{ECG\_Processing} module (Python + NeuroKit2~\cite{makowski2021neurokit2}) reads signals from the PTB-XL dataset, applies peak detection and wavelet analysis, and emits sequence of time stamped events: $R_{peak}$, $P_{onset}$, $P_{offset}$, $QRS_{onset}$, $QRS_{offset}$, $QRS_{onset}$ , and $T_{onset}$, $T_{offset}$. These events are fed into the \texttt{RV\_Monitor} module (Python 2.7 + UPPAAL DBM~\cite{UPPAAL}), which loads the product automaton and, for each event, emits either $c\_True$ or $c\_False$.


%From PTB-XL we gathered a dataset of 30 individuals, each contributing a single ECG recording. Every ECG spans 10 seconds and is sampled at a rate of 100 Hz. As a result, we obtained 3000 data points, where each data point is classified as either currently true or currently false. We evaluate our model using three standard metrics:

From the PTB-XL database~\cite{wagner2020ptb}, we gathered a dataset consisting of ECG recordings from 30 individuals, each contributing a single ECG recording. Each ECG recording spans 10 seconds and is sampled at a rate of 100 Hz. As a result, we obtained 3000 data points per recording, where each data point is classified as either currently true or currently false. We evaluate our model using three standard metrics:

\ignore{\begin{align}
	\text{Accuracy} &= \frac{TP + TN}{TP + TN + FP + FN} \times 100\%, \\
	\text{Sensitivity} &= \frac{TP}{TP + FN} \times 100\%, \\
	\text{Specificity} &= \frac{TN}{TN + FP} \times 100\%,
\end{align}}

accuracy(\%) = (TP + TN)/(TP + TN + FP + FN) $\times$ 100

aensitivity(\%) = [TP/(TP + FN)] $\times$ 100

apecificity(\%) = [TN/(TN + FP)] $\times$ 100


where
	$TP$ denotes true positives,
	$TN$ denotes true negatives,
	$FP$ denotes false positives,
	$FN$ denotes false negatives.


The RV framework achieved an accuracy of $98.3\%$, a sensitivity of $98.6\%$, and a specificity of $98.5\%$.


%\subsubsection{Discussion}
Most ECG classification works, as shown in Table \ref{Table:previous works}, focus on arrhythmia detection and classification. The proposed RV framework efficiently classifies ECG traces with arrhythmia, atrial enlargement, bundle branch block, conduction issues, \etc (cf.~\cite{}). It may be noted that typical ECG traces can deviate from the standard specifications, but with repeated violations of the policies, the user gets an early warning.

%Most ECG classification works, as shown in Table \ref{Table:previous works}.This work considers policies to capture arrhythmia. The proposed RV framework efficiently classifies ECG traces with arrhythmia. It may be noted that typical ECG traces can deviate from the standard specifications, but with repeated violations of the policies, the user gets an early warning.

\begin{table}[]
	\centering
	\caption{\red{Few previous works.}}
	%\vspace{-0.5em}
	\begin{adjustbox}{width=\columnwidth}
	
	{
		\begin{tabular}{|c|c|c|c|c|c|c|}
			\hline
			Author             & \begin{tabular}[c]{@{}c@{}}Type of\\ classification\end{tabular} & Classes                                                    & ECG features                                                                                        & Database & Model & Accuracy \\ \hline
			Vijayavanan et al. \cite{vijayavanan2014automatic}& heart-beat                                                       & \begin{tabular}[c]{@{}c@{}}normal/\\ abnormal\end{tabular} & \begin{tabular}[c]{@{}c@{}}intervals- RR, PR, QT, ST, QRS duration,\\ segments- ST, PR\end{tabular} & MIT-BIH  & PNN   & 96.5     \\ \hline
			Tang and Shu \cite{tang2014classification}      & heart-beat                                                       & \begin{tabular}[c]{@{}c@{}}normal/\\ abnormal\end{tabular} & 3 angle features, 19 temporal features                                                              & MIT-BIH  & QNN   & 91.7     \\ \hline
			Zidelmal et al. \cite{zidelmal2013ecg}    & heart-beat                                                       & \begin{tabular}[c]{@{}c@{}}normal/\\ abnormal\end{tabular} & RR interval, QRS duration                                                                           & MIT-BIH  & SVM   & 98.8     \\ \hline
			Vishwa et al. \cite{vishwa2011clasification}      & heart-beat                                                       & \begin{tabular}[c]{@{}c@{}}normal/\\ abnormal\end{tabular} & RR interval                                                                                         & MIT-BIH  & ANN   & 96.7     \\ \hline
		\end{tabular}
	}
	%	content...
	\end{adjustbox}
	\label{Table:previous works}
	\vspace{-1.5em}
\end{table} 






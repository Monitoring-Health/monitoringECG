%\section{Abstract}
%hypertension paper~\cite{DBLP:journals/esl/PandaAPR24}.
%\\


\section{CAPTURING HEART ABNORMALITIES AS TIMED AUTOMATA}

In the previous section, we discussed different conventional temporal
aspects of an ECG signal. In this section, we discuss the cardiac problems caused by temporal feature deviations and the timed policies
considered to monitor and classify ECG.
As previously stated, a typical ECG signal has various temporal
aspects, including PR, QT, RR, and P-wave intervals, as well as
conventional ranges. Any variations from these normal electrical
patterns can suggest a variety of heart conditions. Different ECG
waves and intervals, their usual ranges, and abnormalities in the heart

when these properties deviate from their safe ranges are presented in
Table \ref{table:abnormal}

\subsection{Capturing cardiac complications due to
	extended PR interval as TA}
	
Regular PR intervals average between 0.12 and 0.20 seconds. A
prolonged PR interval showing delayed transmission of the SA node
impulse to the ventricles is diagnostic of first-degree AV block. Firstdegree heart block is a clinically insignificant finding on its own;
however, heart diseases such as acute rheumatic carditis, an overdose
of digoxin, or an electrolyte imbalance may arise because of it. A
Harvard study found that people with PR intervals longer than 200
ms had an increased risk of atrial fibrillation, pacemaker insertion,
and premature death by roughly a factor of 1.5. Wolff-ParkinsonWhite and Lown-Ganong-Levine syndromes are two conditions in
which a short PR interval is seen because the heart is able to bypass
the AV node delay. The TA in Fig. \ref{fig:TA1} captures the extended PR
interval. The RV monitor raises the alarm whenever the standard PR
interval is violated, indicating the possibility of cardiac abnormalities
due to the prolonged PR interval.



\begin{table*}[]
	\centering
	\caption{ECG features and heart abnormalities}	
	\vspace{-0.5em}
	\resizebox{0.9\linewidth}{!}
	{	
		\begin{tabular}{|c|c|c|c|c|}
			\hline
			
			\begin{tabular}[c]{@{}c@{}}ECG waves\\  and\\ intervals\end{tabular} &
			\begin{tabular}[c]{@{}c@{}}Normal\\  Range\end{tabular} &
			\begin{tabular}[c]{@{}c@{}}Abnormal\\ duration\end{tabular} & Heart Abnormality 
			& References \\ \hline
			\multirow{2}{*}{PR interval}                                         &
			\multirow{2}{*}{0.12-0.20s}                             & $\le$ 0.12 s              
			& Pre-excitation syndromes                 
			& \cite{hampton2019ecg}         \\ \cline{3-5} 
			&                                                         & $\ge$ 0.2 s           
			& First-degree AV block                
			& \cite{hampton2019ecg}          \\ \hline
			\multirow{2}{*}{QRS complex}                                         &
			\multirow{2}{*}{0.08-0.12 s}                            & 0.1 - 0.12 s          
			& Incomplete bundle branch block         
			& \cite{hampton2019ecg}          \\ \cline{3-5} 
			&                                                         & $\ge$ 0.12 s          
			& Complete bundle branch block         
			& \cite{hampton2019ecg}         \\ \hline
			\multirow{2}{*}{QT interval}                                         &
			\multirow{2}{*}{350-480 ms}                             & $\ge$ 480 ms              
			& Abnormal ventricular repolarization                 
			& \cite{hampton2019ecg}        \\ \cline{3-5} 
			&                                                         & $\le$ 390 ms          
			& Abnormal ventricular depolarization or
			repolarization & \cite{hampton2019ecg}         \\ \hline
			RR interval                                                          & 0.6-1.2
			s                                               & variable                      
			& Irregular heart rhythm                           
			& \cite{hampton2019ecg}        \\ \hline
			
			P-wave                                                               & $\leq$
			0.12 s                                        & $\ge$ 0.12 s                               
			& atrial enlargement                                                 
			& \cite{hampton2019ecg}         \\ \hline
		\end{tabular}
	}
	\vspace{-1em}
	\label{table:abnormal}
\end{table*}


\begin{figure}
	\begin{adjustbox}{width=\linewidth}
		
		\begin{tikzpicture}[->,shorten >=1pt,auto,node distance=2.5cm,semithick,initial where=left]
			
			\tikzstyle{every node}=[font=\small]
			
			\tikzstyle{good state}=[circle,thick,draw=blue!75,fill=blue!20,minimum size=5mm]
			\tikzstyle{bad state}=[circle,thick,draw=red!75,fill=red!20,minimum size=3mm,accepting]
			\tikzstyle{dead state}=[rectangle,thick,draw=red!75,fill=red!20,minimum size=5mm]
			
			\node[initial,good state] (N0) {$S_0$};
			\node[good state]         (N1) [right=4.5 of N0] {$S_1$}; %[right of=N0] {$l_1$};
			\node[bad state]        (H) [below=0.75 of N1] {$S_2$};
			\node[bad state]        (H1) [below=0.75 of N0] {$S_3$};
			
			\path (N0) edge  [bend left=20] node [align=center]  {$ P_{onset}, x:=0 $ }( N1)
			
			%edge [loop above] node [align=center] {$ B_{G_1} $: $ b_1=1 $\\$C_{G_1}$: $ c_1=1 $}(N0)  
			edge [loop above] node [align=center] {$ \Sigma \setminus P_{onset} $}(N0)  
			%(H)edge [bend left=20] node [align=center] {$ R, x:=0 $ }(N1)          
			(N1)edge [bend left=0] node [align=center, pos=0.5, below] {$ S_{offset}, x\leq200 $ and $ x\geq120 $} (N0)
			(H1)edge node [align=center] {$ S_{offset}, x\leq200 $\\ and $ x\geq120 $} (N0)
			%(N1) edge  [loop right] node {$ A_{G_1} $} (N1)
			(N1)edge node [align=center] {$ S_{offset}, x>200 $  \\or $ x<120 $} (H)
			edge [loop above] node [align=center] {$ \Sigma \setminus S_{offset} $}(N1)
			%edge [loop above] node [align=center] {$ \Sigma \setminus on $}(H1)
			(H) edge [loop right] node {$ \Sigma \setminus P_{onset} $} (H)
			(H) edge  [bend left=20] node [align=center]  {$ P_{onset}, x:=0 $ }( H1)
			(H1)edge  [bend left=0] node [align=center, pos=0.5, above] {$ S_{offset}, x>200 $  or $ x<120 $} (H)
			edge [loop left] node [align=center] {$ \Sigma \setminus S_{offset} $}(H1);
			
		\end{tikzpicture}
		%content...
	\end{adjustbox}
	\caption{\red{Policy $\varphi_{ECG1}$ specified as TA.}}
	%\caption {figure} {VDTA specifying the constraint: ``\textit{Peer A can undertake a research project only after it has been approved by peer \{B, C\}}".}
	\label{fig:TA1}
\end{figure}


\subsection{ Capturing wide QRS complex as TA}

In general, the QRS-complex ranges between 0.08 and 0.10 seconds. When the time span is between 0.10 and 0.12 seconds, we
classify it as intermediate or prolonged QRS-complex. This may
indicate a left anterior or posterior fascicular block, or an incomplete right or left bundle branch block. When the QRS-complex
lasts longer than 0.12 seconds, medical attention is warranted. An
extended QRS duration is a symptom of a variety of heart rhythm
disorders, such as right bundle branch block, left bundle branch
block, non-specific intraventricular conduction delay, and ventricular arrhythmias such as ventricular tachycardia. Right-sided heart
problems are often indicated by a phenomenon known as right bundle branch block (RBBB). However, the left bundle branch block
(LBBB) is always associated with heart disease, most frequently
in the left ventricle. The prolonged QRS duration is captured by
the TA in Fig. \ref{TA2}. Every time the expected duration is exceeded, the
RV monitor sounds the alarm, warning of the potential for cardiac
problems caused by the prolonged QRS interval.

\begin{figure}
	\begin{adjustbox}{width=\linewidth}
		
		\begin{tikzpicture}[->,shorten >=1pt,auto,node distance=2.5cm,semithick,initial where=left]
			
			\tikzstyle{every node}=[font=\small]
			
			\tikzstyle{good state}=[circle,thick,draw=blue!75,fill=blue!20,minimum size=5mm]
			\tikzstyle{bad state}=[circle,thick,draw=red!75,fill=red!20,minimum size=3mm,accepting]
			\tikzstyle{dead state}=[rectangle,thick,draw=red!75,fill=red!20,minimum size=5mm]
			
			\node[initial,good state] (N0) {$S_0$};
			\node[good state]         (N1) [right=4.5 of N0] {$S_1$}; %[right of=N0] {$l_1$};
			\node[bad state]        (H) [below=0.75 of N1] {$S_2$};
			\node[bad state]        (H1) [below=0.75 of N0] {$S_3$};
			
			\path (N0) edge  [bend left=20] node [align=center]  {$ Q_{onset}, x:=0 $ }( N1)
			
			%edge [loop above] node [align=center] {$ B_{G_1} $: $ b_1=1 $\\$C_{G_1}$: $ c_1=1 $}(N0)  
			edge [loop above] node [align=center] {$ \Sigma \setminus Q_{onset} $}(N0)  
			%(H)edge [bend left=20] node [align=center] {$ R, x:=0 $ }(N1)          
			(N1)edge [bend left=0] node [align=center, pos=0.5, below] {$ S_{offset}, x\leq120 $ and $ x\geq80 $} (N0)
			(H1)edge node [align=center] {$ S_{offset}, x\leq120 $\\ and $ x\geq80 $} (N0)
			%(N1) edge  [loop right] node {$ A_{G_1} $} (N1)
			(N1)edge node [align=center] {$ S_{offset}, x>120 $  \\or $ x<80 $} (H)
			edge [loop above] node [align=center] {$ \Sigma \setminus S_{offset} $}(N1)
			%edge [loop above] node [align=center] {$ \Sigma \setminus on $}(H1)
			(H) edge [loop right] node {$ \Sigma \setminus Q_{onset} $} (H)
			(H) edge  [bend left=20] node [align=center]  {$ Q_{onset}, x:=0 $ }( H1)
			(H1)edge  [bend left=0] node [align=center, pos=0.5, above] {$ S_{offset}, x>120 $  or $ x<80 $} (H)
			edge [loop left] node [align=center] {$ \Sigma \setminus S_{offset} $}(H1);
			
		\end{tikzpicture}
		%content...
	\end{adjustbox}
	\caption{\red{Policy $\varphi_{ECG1}$ specified as TA.}}
	%\caption {figure} {VDTA specifying the constraint: ``\textit{Peer A can undertake a research project only after it has been approved by peer \{B, C\}}".}
	\label{fig:TA2}
\end{figure}

\subsection{Capturing prolonged QT interval as TA}

The QTc interval should be between 350 and 480 milliseconds. It is
prolonged in people with certain electrolyte disorders, and it is also
prolonged by some drugs. Ventricular tachycardia can be caused
by a prolonged QT interval (more than 480 ms). Long QT Interval
Syndromes occur when the QTc is larger than 480 ms (LQTS). This
condition has significant clinical implications because it typically suggests a higher risk of malignant ventricular arrhythmia, syncope,
and sudden death. Short QTc syndrome occurs when QTc is less than
0.35 s and can result in hypercalcemia and malignant arrhythmia.
The elongated QT interval is depicted by the TA in Fig. \ref{TA3}. When
the typical QT interval is exceeded, the RV monitor sounds an alert
because this may be a sign of cardiac problems.

\begin{figure}
	\begin{adjustbox}{width=\linewidth}
		
		\begin{tikzpicture}[->,shorten >=1pt,auto,node distance=2.5cm,semithick,initial where=left]
			
			\tikzstyle{every node}=[font=\small]
			
			\tikzstyle{good state}=[circle,thick,draw=blue!75,fill=blue!20,minimum size=5mm]
			\tikzstyle{bad state}=[circle,thick,draw=red!75,fill=red!20,minimum size=3mm,accepting]
			\tikzstyle{dead state}=[rectangle,thick,draw=red!75,fill=red!20,minimum size=5mm]
			
			\node[initial,good state] (N0) {$S_0$};
			\node[good state]         (N1) [right=4.5 of N0] {$S_1$}; %[right of=N0] {$l_1$};
			\node[bad state]        (H) [below=0.75 of N1] {$S_2$};
			\node[bad state]        (H1) [below=0.75 of N0] {$S_3$};
			
			\path (N0) edge  [bend left=20] node [align=center]  {$ P_{onset}, x:=0 $ }( N1)
			
			%edge [loop above] node [align=center] {$ B_{G_1} $: $ b_1=1 $\\$C_{G_1}$: $ c_1=1 $}(N0)  
			edge [loop above] node [align=center] {$ \Sigma \setminus P_{onset} $}(N0)  
			%(H)edge [bend left=20] node [align=center] {$ R, x:=0 $ }(N1)          
			(N1)edge [bend left=0] node [align=center, pos=0.5, below] {$ S_{offset}, x\leq200 $ and $ x\geq120 $} (N0)
			(H1)edge node [align=center] {$ S_{offset}, x\leq200 $\\ and $ x\geq120 $} (N0)
			%(N1) edge  [loop right] node {$ A_{G_1} $} (N1)
			(N1)edge node [align=center] {$ S_{offset}, x>200 $  \\or $ x<120 $} (H)
			edge [loop above] node [align=center] {$ \Sigma \setminus S_{offset} $}(N1)
			%edge [loop above] node [align=center] {$ \Sigma \setminus on $}(H1)
			(H) edge [loop right] node {$ \Sigma \setminus P_{onset} $} (H)
			(H) edge  [bend left=20] node [align=center]  {$ P_{onset}, x:=0 $ }( H1)
			(H1)edge  [bend left=0] node [align=center, pos=0.5, above] {$ S_{offset}, x>200 $  or $ x<120 $} (H)
			edge [loop left] node [align=center] {$ \Sigma \setminus S_{offset} $}(H1);
			
		\end{tikzpicture}
		%content...
	\end{adjustbox}
	\caption{\red{Policy $\varphi_{ECG1}$ specified as TA.}}
	%\caption {figure} {VDTA specifying the constraint: ``\textit{Peer A can undertake a research project only after it has been approved by peer \{B, C\}}".}
	\label{fig:TA3}
\end{figure}


\subsection{Capturing wide P-wave as TA}

In addition, when the P-wave duration is longer than normal, it
usually indicates that one or both atria are enlarged (hypertrophied).
P waves can broaden or amplify as a result of atrial enlargements.
The shape of the P waves can be altered by ectopic atrial beats.
An absence of P waves is a characteristic of many abnormal heart
rhythms, such as atrial fibrillation and junctional arrhythmias. Short
RP can occur when the P waves are obscured by the end of the QRS
complex, as can happen in atrioventricular reentrant tachycardia.
The wide P-wave interval is depicted by the TA in the Fig. \ref{TA4}. When
the specified duration is exceeded, the RV monitor will sound the
alarm to alert the user to the fact that there is a potential for cardiac
irregularities as a result of the longer P-wave interval.

\begin{figure}
	\begin{adjustbox}{width=\linewidth}
		
		\begin{tikzpicture}[->,shorten >=1pt,auto,node distance=2.5cm,semithick,initial where=left]
			
			\tikzstyle{every node}=[font=\small]
			
			\tikzstyle{good state}=[circle,thick,draw=blue!75,fill=blue!20,minimum size=5mm]
			\tikzstyle{bad state}=[circle,thick,draw=red!75,fill=red!20,minimum size=3mm,accepting]
			\tikzstyle{dead state}=[rectangle,thick,draw=red!75,fill=red!20,minimum size=5mm]
			
			\node[initial,good state] (N0) {$S_0$};
			\node[good state]         (N1) [right=4.5 of N0] {$S_1$}; %[right of=N0] {$l_1$};
			\node[bad state]        (H) [below=0.75 of N1] {$S_2$};
			\node[bad state]        (H1) [below=0.75 of N0] {$S_3$};
			
			\path (N0) edge  [bend left=20] node [align=center]  {$ P_{onset}, x:=0 $ }( N1)
			
			%edge [loop above] node [align=center] {$ B_{G_1} $: $ b_1=1 $\\$C_{G_1}$: $ c_1=1 $}(N0)  
			edge [loop above] node [align=center] {$ \Sigma \setminus P_{onset} $}(N0)  
			%(H)edge [bend left=20] node [align=center] {$ R, x:=0 $ }(N1)          
			(N1)edge [bend left=0] node [align=center, pos=0.5, below] {$ S_{offset}, x\leq200 $ and $ x\geq120 $} (N0)
			(H1)edge node [align=center] {$ S_{offset}, x\leq200 $\\ and $ x\geq120 $} (N0)
			%(N1) edge  [loop right] node {$ A_{G_1} $} (N1)
			(N1)edge node [align=center] {$ S_{offset}, x>200 $  \\or $ x<120 $} (H)
			edge [loop above] node [align=center] {$ \Sigma \setminus S_{offset} $}(N1)
			%edge [loop above] node [align=center] {$ \Sigma \setminus on $}(H1)
			(H) edge [loop right] node {$ \Sigma \setminus P_{onset} $} (H)
			(H) edge  [bend left=20] node [align=center]  {$ P_{onset}, x:=0 $ }( H1)
			(H1)edge  [bend left=0] node [align=center, pos=0.5, above] {$ S_{offset}, x>200 $  or $ x<120 $} (H)
			edge [loop left] node [align=center] {$ \Sigma \setminus S_{offset} $}(H1);
			
		\end{tikzpicture}
		%content...
	\end{adjustbox}
	\caption{\red{Policy $\varphi_{ECG1}$ specified as TA.}}
	%\caption {figure} {VDTA specifying the constraint: ``\textit{Peer A can undertake a research project only after it has been approved by peer \{B, C\}}".}
	\label{fig:TA4}
\end{figure}

\subsection{Capturing variation in heartbeat as TA}

Variations in the RR interval have been extensively researched
in the context of irregular cardiac rhythm. The fluctuations in the
time intervals between individual heartbeats (R-peaks) are measured
by heart rate variability (HRV). The HRV can provide insights into
autonomic neural function as well as sympathetic-parasympathetic
autonomic balance and cardiovascular health ([13]). The extended
RR interval is captured by the TA in Fig. \ref{TA5}. Every time the normal
RR interval is violated, the RV monitor sounds the alarm, potentially
alerting the patient to the likelihood of cardiac problems caused by
the prolonged RR interval.


\begin{figure}
	\begin{adjustbox}{width=\linewidth}
		
		\begin{tikzpicture}[->,shorten >=1pt,auto,node distance=2.5cm,semithick,initial where=left]
			
			\tikzstyle{every node}=[font=\small]
			
			\tikzstyle{good state}=[circle,thick,draw=blue!75,fill=blue!20,minimum size=5mm]
			\tikzstyle{bad state}=[circle,thick,draw=red!75,fill=red!20,minimum size=3mm,accepting]
			\tikzstyle{dead state}=[rectangle,thick,draw=red!75,fill=red!20,minimum size=5mm]
			
			\node[initial,good state] (N0) {$S_0$};
			\node[good state]         (N1) [right=4.5 of N0] {$S_1$}; %[right of=N0] {$l_1$};
			\node[bad state]        (H) [below=0.75 of N1] {$S_2$};
			\node[bad state]        (H1) [below=0.75 of N0] {$S_3$};
			
			\path (N0) edge  [bend left=20] node [align=center]  {$ P_{onset}, x:=0 $ }( N1)
			
			%edge [loop above] node [align=center] {$ B_{G_1} $: $ b_1=1 $\\$C_{G_1}$: $ c_1=1 $}(N0)  
			edge [loop above] node [align=center] {$ \Sigma \setminus P_{onset} $}(N0)  
			%(H)edge [bend left=20] node [align=center] {$ R, x:=0 $ }(N1)          
			(N1)edge [bend left=0] node [align=center, pos=0.5, below] {$ S_{offset}, x\leq200 $ and $ x\geq120 $} (N0)
			(H1)edge node [align=center] {$ S_{offset}, x\leq200 $\\ and $ x\geq120 $} (N0)
			%(N1) edge  [loop right] node {$ A_{G_1} $} (N1)
			(N1)edge node [align=center] {$ S_{offset}, x>200 $  \\or $ x<120 $} (H)
			edge [loop above] node [align=center] {$ \Sigma \setminus S_{offset} $}(N1)
			%edge [loop above] node [align=center] {$ \Sigma \setminus on $}(H1)
			(H) edge [loop right] node {$ \Sigma \setminus P_{onset} $} (H)
			(H) edge  [bend left=20] node [align=center]  {$ P_{onset}, x:=0 $ }( H1)
			(H1)edge  [bend left=0] node [align=center, pos=0.5, above] {$ S_{offset}, x>200 $  or $ x<120 $} (H)
			edge [loop left] node [align=center] {$ \Sigma \setminus S_{offset} $}(H1);
			
		\end{tikzpicture}
		%content...
	\end{adjustbox}
	\caption{\red{Policy $\varphi_{ECG1}$ specified as TA.}}
	%\caption {figure} {VDTA specifying the constraint: ``\textit{Peer A can undertake a research project only after it has been approved by peer \{B, C\}}".}
	\label{fig:TA5}
\end{figure}









